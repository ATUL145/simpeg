\documentclass[]{article}

\renewcommand{\div}{\nabla\cdot\,}
\newcommand{\grad}{\vec \nabla}
\newcommand{\curl}{{\vec \nabla}\times\,}
\newcommand {\J}{{\vec J}}
\renewcommand{\H}{{\vec H}}
\newcommand {\E}{{\vec E}}
\newcommand{\dcurl}{{\mathbf C}}
\newcommand{\dgrad}{{\mathbf G}}
\newcommand{\Acf}{{\mathbf A_c^f}}
\newcommand{\Ace}{{\mathbf A_c^e}}
\renewcommand{\S}{{\mathbf \Sigma}}
\newcommand{\St}{{\mathbf \Sigma_\tau}}
\newcommand{\T}{{\mathbf T}}
\newcommand{\Tt}{{\mathbf T_\tau}}
\newcommand{\diag}[1]{\,{\sf diag}\left( #1 \right)}

%Common mass matricies
\newcommand{\M}{{\mathbf M}}
\newcommand{\MfMui}{{\M^f_{\mu^{-1}}}}
\newcommand{\MeSig}{{\M^e_\sigma}}
\newcommand{\MeSigInf}{{\M^e_{\sigma_\infty}}}
\newcommand{\MeSigO}{{\M^e_{\sigma_0}}}
\newcommand{\Me}{{\M^e}}
\newcommand{\Mes}[1]{{\M^e_{#1}}}
\newcommand{\Mee}{{\M^e_e}}
\newcommand{\Mej}{{\M^e_j}}

\newcommand{\BigO}[1]{\mathcal{O}\bigl(#1\bigr)}

% ********** TDIP paper

\newcommand{\bE}{\mathbf{E}}
\newcommand{\bH}{\mathbf{H}}
\newcommand{\B}{\vec{B}}
\newcommand{\D}{\vec{D}}
\renewcommand{\H}{\vec{H}}
\newcommand{\s}{\vec{s}}
\newcommand{\bfJ}{\bf{J}}
\newcommand{\vecm}{\vec m}
\renewcommand{\Re}{\mathsf{Re}}
\renewcommand{\Im}{\mathsf{Im}}

\renewcommand {\j}  { {\vec j} }
\newcommand {\h}  { {\vec h} }
\renewcommand {\b}  { {\vec b} }
\newcommand {\e}  { {\vec e} }
\renewcommand {\d}  { {\vec d} }
\renewcommand {\u}  { {\vec u} }

\newcommand{\I}{\vec{I}}


\usepackage{pslatex,palatino,avant,graphicx,color,amsmath}
% \usepackage[margin=2cm]{geometry}

\begin{document}
\title{TEM}

\section{Sensitivity Calculation}

\begin{subequations}
    \begin{align}
        \dcurl \e^{(t+1)} + \frac{\b^{(t+1)} - \b^{(t)}}{\delta t} = 0 \\
        \dcurl^\top \MfMui \b^{(t+1)} - \MeSig \e^{(t+1)} = \Me \j_s^{(t+1)}
    \end{align}
\end{subequations}

Using Gauss-Newton to solve the inverse problem requires the ability to calculate the product of the Jacobian and a vector, as well as the transpose of the Jacobian times a vector. The above system can be rewritten as

\begin{align}
    \mathbf{A} \u^{(t+1)} + \mathbf{B} \u^{(t)}= \s^{(t+1)}
\end{align}
where
\begin{subequations}
    \begin{align}
        \mathbf{A} =
        \left[
            \begin{array}{cc}
                \frac{1}{\delta t} \mathbf{I} & \dcurl \\
                \dcurl^\top \MfMui & -\MeSig
            \end{array}
        \right] \\
        \mathbf{B} =
        \left[
            \begin{array}{cc}
                -\frac{1}{\delta t} \mathbf{I} & 0 \\
                0 & 0
            \end{array}
        \right] \\
        \u^{(k)} = \left[
        \begin{array}{c}
            \b^{(k)}\\
            \e^{(k)}
        \end{array}
        \right] \\
        \s^{(k)} = \left[
        \begin{array}{c}
            0\\
            \Me \j^{(k)}_s
        \end{array}
        \right]
    \end{align}
\end{subequations}

The entire time dependent system can be written in a single matrix expression
\begin{align}
    \hat{\mathbf{A}} \hat{u} = \hat{s}
\end{align}
where
\begin{subequations}
    \begin{align}
        \mathbf{\hat{A}} = \left[
        \begin{array}{cccc}
            A & 0 & & \\
            B & A & & \\
              & \ddots & \ddots & \\
              & & B & A
        \end{array}
        \right] \\
        \hat{u} = \left[
            \begin{array}{c}
                \u^{(1)} \\
                \u^{(2)} \\
                \vdots \\
                \u^{(N)}
            \end{array} \right]\\
        \hat{s} = \left[
            \begin{array}{c}
                \s^{(1)} - \mathbf{B} \u^{(0)} \\
                \s^{(2)} \\
                \vdots \\
                \s^{(N)}
            \end{array}
        \right]
    \end{align}
\end{subequations}

For the fields $\u$, the measured data is given by
\begin{align}
    \vec{d} = \mathbf{Q} \u
\end{align}
The sensitivity matrix $\mathbf{J}$ is then defined as
\begin{align}
    \mathbf{J} = \mathbf{Q} \frac{\partial \u}{\partial \sigma}
\end{align}


Defining the function $\vec{c}(m,\vec{u})$ to be
\begin{align}
    \vec{c}(m,\u) = \hat{\mathbf{A}} \vec{u} - \vec{q} = \vec{0}
\end{align}
then
\begin{align}
    \frac{\partial \vec{c}}{\partial m} \partial m
    + \frac{\partial \vec{c}}{\partial \u} \partial \vec{u} = 0
\end{align}
or
\begin{align}
    \frac{\partial \vec{u}}{\partial m} = -\left(\frac{\partial \vec{c}}{\partial \u} \right)^{-1} \frac{\partial \vec{c}}{\partial m}
\end{align}


Differentiating, we find that
\begin{align}
    \frac{\partial \vec{c}}{\partial \hat{u}} = \hat{\mathbf{A}}
\end{align}
and
\begin{align}
    \frac{\partial \vec{c}}{\partial \sigma} = \mathbf{G}_\sigma =
    \left[
        \begin{array}{c}
            g_\sigma^{(1)}\\
            g_\sigma^{(2)}\\
            \vdots \\
            g_\sigma^{(N)}
        \end{array}
    \right]
\end{align}
with
\begin{subequations}
    \begin{align}
        g_\sigma^{(n)} =
        \left[
            \begin{array}{c}
                \mathbf{0} \\
                - \diag{\e^{(n)}} \Ace \diag{\vec{V}}
            \end{array}
        \right]
    \end{align}
\end{subequations}

\subsection{Implementing $\mathbf{J}$ times a vector}
Multiplying $\mathbf{J}$ onto a vector can be broken into three steps
\begin{enumerate}
\item Compute $\vec{p} = \mathbf{G}m$
\item Solve $\hat{\mathbf{A}} \vec{y} = \vec{p}$
\item Compute $\vec{w} = -\mathbf{Q} \vec{y}$
\end{enumerate}

\begin{subequations}
    \begin{align}
        \vec{p}^{(n)} = \left[
            \begin{array}{c}
                0 \\
                \vec{p}_e^{(n)}
            \end{array}
        \right] \\
        \vec{p}_e^{(n)} = - \diag{\e^{(n)}} \Ace \diag{V} m
    \end{align}
\end{subequations}

\paragraph{First time step}

\begin{subequations}
    \begin{align}
        \dcurl \vec{y}_{e}^{(1)} + \frac{1}{\delta t} \vec{y}_{b}^{(1)} = 0 \\
        \dcurl^\top \MfMui \vec{y}_b^{(1)} - \MeSig \vec{y}_e^{(1)} = \vec{p}_e^{(1)}
    \end{align}
\end{subequations}

\begin{subequations}
    \begin{align}
        \left( \MfMui \dcurl \MeSig^{-1} \dcurl^\top \MfMui + \frac{1}{\delta t} \MfMui \right) \vec{y}_{b}^{(1)} = \MfMui \dcurl \MeSig^{-1} \vec{p}_e^{(1)} \\
        \vec{y}_e^{(1)} = \MeSig^{-1} \dcurl^\top \MfMui \vec{y}_b^{(1)} - \MeSig^{-1} \vec{p}_e^{(1)}
    \end{align}
\end{subequations}

\paragraph{Remaining time steps}

\begin{subequations}
    \begin{align}
        \dcurl \vec{y}_{e}^{(t+1)} + \frac{1}{\delta t} \vec{y}_{b}^{(t+1)}
        {\color{red}- \frac{1}{\delta t} \vec{y}_{b}^{(t)} }
        = 0 \\
        \dcurl^\top \MfMui \vec{y}_b^{(t+1)} - \MeSig \vec{y}_e^{(t+1)} = \vec{p}_e^{(t+1)}
    \end{align}
\end{subequations}

\begin{subequations}
        \begin{align}
            \left( \MfMui \dcurl \MeSig^{-1} \dcurl^\top \MfMui + \frac{1}{\delta t} \MfMui \right) \vec{y}_{b}^{(t+1)} =
            {\color{red} \frac{1}{\delta t} \MfMui \vec{y}_b^{(t)} }
            + \MfMui \dcurl \MeSig^{-1} \vec{p}_e^{(t+1)} \\
            \vec{y}_e^{(t+1)} = \MeSig^{-1} \dcurl^\top \MfMui \vec{y}_b^{(t+1)} - \MeSig^{-1} \vec{p}_e^{(t+1)}
        \end{align}
    \end{subequations}

\subsection{Implementing $\mathbf{J}^\top$ onto a vector}



\end{document}
